\documentclass[11pt,a4paper]{article}
\usepackage[utf8]{inputenc}
\usepackage[spanish]{babel}
%\usepackage{amsmath}
%\usepackage{amsfonts}
%\usepackage{amssymb}
\usepackage{hyperref}
\usepackage[pdftex]{graphicx}
\usepackage{fancyhdr}
\usepackage[font=small,labelfont=bf]{caption}
\pagestyle{fancy}
\lhead{\bfseries Redes de Datos -- IPv6}
\rhead{}
%\chead{}
\newcommand{\HRule}{\rule{\linewidth}{0.5mm}}
\usepackage[left=2cm,right=2cm,top=2cm,bottom=2cm]{geometry}
\author{Ignacio Perez Laborda}

\hypersetup{pdfborder = {0 0 0}}

\begin{document}

\begin{titlepage}

\begin{center}

% Upper part of the page
\includegraphics[width=0.25\textwidth]{./logo_UB.png}\\[1cm] 

\textsc{\LARGE Universidad de Belgrano}\\[1.5cm]

\textsc{\Large Redes De Datos}\\[0.5cm]

%title
\HRule \\[0.4cm]
{ \huge \bfseries Trabajo Practico 2 -- Cloud Computing}\\[0.4cm]
\HRule \\[1.5cm]

% Author and supervisor
\begin{minipage}{0.4\textwidth}
\begin{flushleft} \large
\emph{Alumno:}\\
Ignacio \textsc{P\'erez Laborda}\\
Barbara \textsc{Mart\'inez}\\
\end{flushleft}
\end{minipage}
\begin{minipage}{0.4\textwidth}
\begin{flushright} \large
\emph{Matricula:} \\
502--10426\\
502--10402\\
\end{flushright}
\end{minipage}\\[1.5cm]

\vfill

%Bottom of the page
{\large \today}

\end{center}

\end{titlepage}


\tableofcontents

\listoffigures

\newpage

\section{¿Que es IP?}

\section{El estandar IPv4}

\section{Porque abandonar IPv4}
Uno de los mayores problemas que presenta IPv4 es su tamaño, el cual ya ha quedado chico para los
estándares actuales. Este es un numero de solo 32bits con lo cual la cantidad máxima de números
posibles es $2^{32}$ cual da un total de 4.294.967.296 posibles dirección eso es menos de una
dirección por habitante\footnote{población actual de 7 mil millones de personas}.\par
Esto se ve mas limitante con la llegada masiva de los dispositivos mobiles, los cuales se conectan
a internet y la entrada de los grandes mercados emergentes del BRIC (Brasil, Rusia, India, China).

\section{La llegada de IPv6}
\section{Implementar IPv6}
\section{Desarrollo a futuro}

\newpage
\begin{thebibliography}{9}
%esta se pone de ejemplo de como se deben hacer las bibliografias, despues se debe borrar
\bibitem{wnaReport1}
  WNA
  \emph{Comparison of Lifecycle Greenhouse Gas Emissions of Various Electricity Generation Sources}.
  London, UK,
  2011.
		
		
\bibitem{poolIPv4}
 nro.net
 \emph{Free Pool of IPv4 Address Space Depleted.} 
 19 Mar 2013\\
	\url{https://www.nro.net/news/ipv4-free-pool-depleted}		
		
\end{thebibliography}

\end{document}
