\documentclass[11pt,a4paper]{article}
\usepackage[utf8]{inputenc}
\usepackage[spanish]{babel}
%\usepackage{amsmath}
%\usepackage{amsfonts}
%\usepackage{amssymb}
\usepackage{wrapfig}
\usepackage{float}
\usepackage{hyperref}
\usepackage[pdftex]{graphicx}
\usepackage{fancyhdr}
\usepackage[font=small,labelfont=bf]{caption}
\pagestyle{fancy}
\lhead{\bfseries Redes de Datos -- IPv6}
\rhead{}
%\chead{}
\newcommand{\HRule}{\rule{\linewidth}{0.5mm}}
\usepackage[left=2cm,right=2cm,top=2cm,bottom=2cm]{geometry}
\author{Ignacio Perez Laborda}

\renewcommand{\thefootnote}{\roman{footnote}}

\hypersetup{pdfborder = {0 0 0}}

\begin{document}

\begin{titlepage}

\begin{center}

% Upper part of the page
\includegraphics[width=0.25\textwidth]{./logo_UB.png}\\[1cm] 

\textsc{\LARGE Universidad de Belgrano}\\[1.5cm]

\textsc{\Large Redes De Datos}\\[0.5cm]

%title
\HRule \\[0.4cm]
{ \huge \bfseries Trabajo Practico 2 -- Cloud Computing}\\[0.4cm]
\HRule \\[1.5cm]

% Author and supervisor
\begin{minipage}{0.4\textwidth}
\begin{flushleft} \large
\emph{Alumno:}\\
Ignacio \textsc{P\'erez Laborda}\\
Barbara \textsc{Mart\'inez}\\
\end{flushleft}
\end{minipage}
\begin{minipage}{0.4\textwidth}
\begin{flushright} \large
\emph{Matricula:} \\
502--10426\\
502--10402\\
\end{flushright}
\end{minipage}\\[1.5cm]

\vfill

%Bottom of the page
{\large \today}

\end{center}

\end{titlepage}


\tableofcontents

\listoffigures

\newpage

\section{¿Que es IP?}
\subsection{Introducción Histórica}
En plena guerra fría para fines de los años sesenta, el Departamento de Defensa (DoD) norteamericano
necesitaba de un nuevo método para conectar sus distintos centros de investigación y centros 
gubernamentales y que además pueda ser extendida a otros medios de propagación. Esta responsabilidad 
cayo en manos de de ARPA(Advanced Research Projects Agency), la cual en la
reunión de la ACM de 1967 se diseño su estructura básica y se la nombro ARPANET. Para el año 
siguiente se realizaron las primeras adjudicaciones para la implementación de una red de conmutación
de paquetes con una velocidad de 50kbps\footnote{kilobits por segundo}. Dicha implementación fue
adjudicada a la recientemente creada BBN (Bolt Beranek and Newman), la cual fue creada para ese
propósito.\par
Para el año 1969 fue instalado el primer nodo de ARPANET en la Universidad de California en Los
Angeles seguida por los nodos de Stanford, Santa Barbara y la Universidad de Utah. Con estos cuatro
nodos se dio inicio a la red que al cabo de dos años ya se había expandido por todo USA y dos años
después cruzaría el atlántico para desembarcar en Europa.\par
Corriendo el año 1974 ARPANET ya necesitaba de un nuevo protocolo ya que el usado hasta el momento, 
llamado NCP (Network Control Protocol) mostraba muchas falencias al crecer tanto la red. La solución
vino de la mano de un paper que publico Vinton Cerf y Robert Kahn. Este protocolo fue el 
TCP(Transmission Control Protocol), pero este tambien se enfrento a ciertos problemas y 
restricciones.
\begin{wrapfigure}{r}{0.5\textwidth}
\centering
  \includegraphics[width=0.48\textwidth]{historiaTCPIP.png}
 \caption[Historia de TCP/IP]{Linea temporal de TCP/IP}
\vspace{-15pt}
\end{wrapfigure}
Para el año 1978 se decidió dividir las responsabilidades entre un par de protocolos; el nuevo IP
(Internet Protocol) que se encargaría de enrutar los paquetes y comunicaciones de dispositivo a 
dispositivo, y TCP para la comunicación confiable de host a host. Aunque son dos protocolos que 
trabajan en capas distintas fueron pensados para operar en conjunto dentro de un conjunto, por eso
comúnmente se los denomina TCP/IP. La version actual, numero 4, del susodicho protocolo y de la cual
hablaremos mas adelante fue especificada en el año 1979.\par
Ya en la siguiente década ARPANET migro completamente a TCP/IP por mandato del Departamento de 
Defensa. En ese mismo año 1983 la red fue divida en dos ARPANET que siguió con su alcance original y
MILNET donde se desempeñarían las comunicaciones militares. Pero otra gran suceso de ese año fue la
inclusión del protocolo en el UNIX de la Universidad de Berkeley, el BSD 4.2.\par
Saltando tres años situándonos en 1986, la National Science Foundation construyo lo que serian los
cimientos de la internet que conocemos al financiar la construcción de una red para la conexión de
sus supercomputadoras. Esto gracias a la apertura de la NSF permitió a particulares conectarse a 
dicha red lo cual ayudo a su gran crecimiento, todo esto estaba y esta apoyado por el TCP/IP. Esta 
red existió hasta el año 1993, para ese entonces se era lo suficientemente maduro y rentable como 
para ser comercial y ya existían empresas dispuestas y con el conocimiento para llevarlo a cabo. Se 
puso en marcha un plan al año siguiente para reducir la influencia de la NFS y aumentar la 
rentabilidad de los incipientes ISP\footnote{Internet Service Provider} privados.\par
Mientras tanto el DoD y el gobierno norteamericano eligieron adoptar el modelo OSI y se pensó que
esto supondría el fin del modelo TCP/IP, incluso el gobierno llego a obligar su uso masivo, pero
TCP/IP siguió evolucionando sobre el fundamento de la practica y su calidad de estándar abierto.\par
Lo que nos puede dejar esta muy resumida historia del protocolo es que la internet no tiene un claro
inventor ni un destino claro, sino fue el trabajo en conjunto de varias personas e instituciones las
cuales querían solucionar sus problemas de comunicación. Fue gracias a las soluciones abiertas las
cuales permitió el mejoramiento por los usuarios, también los niveles de abstracción que se 
permitieron hizo realidad que varias tecnologías puedan convivir en armonía. Esta filosofía abierta
tan característica y que ha marcado tanto a la industria puede resumirse en el lema de la IETF\footnote{Internet Engineering Task Force}, expresado por David Clark: "Nosotros rechazamos reyes, presidentes y votaciones. Nosotros creemos en el consenso y en el código funcional".


\section{El estándar IPv4}

\section{Porque abandonar IPv4}
Uno de los mayores problemas que presenta IPv4 es su tamaño, el cual ya ha quedado chico para los
estándares actuales. Este es un numero de solo 32bits con lo cual la cantidad máxima de números
posibles es $2^{32}$ cual da un total de 4.294.967.296 posibles dirección eso es menos de una
dirección por habitante\footnote{población actual de 7 mil millones de personas}.\par
Esto se ve mas limitante con la llegada masiva de los dispositivos mobiles, los cuales se conectan
a internet y la entrada de los grandes mercados emergentes del BRIC (Brasil, Rusia, India, China).

\section{La llegada de IPv6}
\section{Implementar IPv6}
\section{Desarrollo a futuro}

\newpage
\begin{thebibliography}{9}
%esta se pone de ejemplo de como se deben hacer las bibliografias, despues se debe borrar
\bibitem{TaoIETF}
  Paul Hoffman
  \emph{The Tao of IETF: A Novice's Guide to the Internet Engineering Task Force}.
  USA,
  2012.
		
\bibitem{historyTCPiP}
  Gary C. Kessler
  \emph{An Overview of TCP/IP Protocols and the Internet}.
  USA,
  9 Nov 2010.
		
\bibitem{poolIPv4}
 nro.net
 \emph{Free Pool of IPv4 Address Space Depleted.} 
 19 Mar 2013\\
	\url{https://www.nro.net/news/ipv4-free-pool-depleted}		
		
\end{thebibliography}

\end{document}
