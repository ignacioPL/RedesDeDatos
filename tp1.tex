\documentclass[11pt,a4paper]{article}
\usepackage[utf8]{inputenc}
\usepackage[spanish]{babel}
%\usepackage{amsmath}
%\usepackage{amsfonts}
%\usepackage{amssymb}
\usepackage{wrapfig}
\usepackage{float}
\usepackage{hyperref}
\usepackage[pdftex]{graphicx}
\usepackage{fancyhdr}
\usepackage[font=small,labelfont=bf]{caption}
\pagestyle{fancy}
\lhead{\bfseries Redes de Datos -- IPv6}
\rhead{}
%\chead{}
\newcommand{\HRule}{\rule{\linewidth}{0.5mm}}
\usepackage[left=2cm,right=2cm,top=2cm,bottom=2cm]{geometry}
\author{Ignacio Perez Laborda}

\renewcommand{\thefootnote}{\roman{footnote}}

\hypersetup{pdfborder = {0 0 0}}

\begin{document}

\begin{titlepage}

\begin{center}

% Upper part of the page
\includegraphics[width=0.25\textwidth]{./logo_UB.png}\\[1cm] 

\textsc{\LARGE Universidad de Belgrano}\\[1.5cm]

\textsc{\Large Redes De Datos}\\[0.5cm]

%title
\HRule \\[0.4cm]
{ \huge \bfseries Trabajo Practico 2 -- Cloud Computing}\\[0.4cm]
\HRule \\[1.5cm]

% Author and supervisor
\begin{minipage}{0.4\textwidth}
\begin{flushleft} \large
\emph{Alumno:}\\
Ignacio \textsc{P\'erez Laborda}\\
Barbara \textsc{Mart\'inez}\\
\end{flushleft}
\end{minipage}
\begin{minipage}{0.4\textwidth}
\begin{flushright} \large
\emph{Matricula:} \\
502--10426\\
502--10402\\
\end{flushright}
\end{minipage}\\[1.5cm]

\vfill

%Bottom of the page
{\large \today}

\end{center}

\end{titlepage}


\tableofcontents

\listoffigures

\newpage

\section{¿Que es IP?}
\subsection{Introducción Histórica}
En plena guerra fría para fines de los años sesenta, el Departamento de Defensa (DoD) norteamericano
necesitaba de un nuevo método para conectar sus distintos centros de investigación y centros 
gubernamentales y que además pueda ser extendida a otros medios de propagación. Esta responsabilidad 
cayo en manos de de ARPA(Advanced Research Projects Agency), la cual en la
reunión de la ACM de 1967 se diseño su estructura básica y se la nombro ARPANET. Para el año 
siguiente se realizaron las primeras adjudicaciones para la implementación de una red de conmutación
de paquetes con una velocidad de 50kbps\footnote{kilobits por segundo}. Dicha implementación fue
adjudicada a la recientemente creada BBN (Bolt Beranek and Newman), la cual fue creada para ese
propósito.\par
Para el año 1969 fue instalado el primer nodo de ARPANET en la Universidad de California en Los
Angeles seguida por los nodos de Stanford, Santa Barbara y la Universidad de Utah. Con estos cuatro
nodos se dio inicio a la red que al cabo de dos años ya se había expandido por todo USA y dos años
después cruzaría el atlántico para desembarcar en Europa.\par
Corriendo el año 1974 ARPANET ya necesitaba de un nuevo protocolo ya que el usado hasta el momento, 
llamado NCP (Network Control Protocol) mostraba muchas falencias al crecer tanto la red. La solución
vino de la mano de un paper que publico Vinton Cerf y Robert Kahn. Este protocolo fue el 
TCP(Transmission Control Protocol), pero este también se enfrento a ciertos problemas y 
restricciones.
\begin{wrapfigure}{r}{0.5\textwidth}
\centering
  \includegraphics[width=0.48\textwidth]{historiaTCPIP.png}
 \caption[Historia de TCP/IP]{Linea temporal de TCP/IP}
\vspace{-15pt}
\end{wrapfigure}
Para el año 1978 se decidió dividir las responsabilidades entre un par de protocolos; el nuevo IP
(Internet Protocol) que se encargaría de enrutar los paquetes y comunicaciones de dispositivo a 
dispositivo, y TCP para la comunicación confiable de host a host. Aunque son dos protocolos que 
trabajan en capas distintas fueron pensados para operar en conjunto dentro de un conjunto, por eso
comúnmente se los denomina TCP/IP. La version actual, numero 4, del susodicho protocolo y de la cual
hablaremos mas adelante fue especificada en el año 1979.\par
Ya en la siguiente década ARPANET migro completamente a TCP/IP por mandato del Departamento de 
Defensa. En ese mismo año 1983 la red fue divida en dos ARPANET que siguió con su alcance original y
MILNET donde se desempeñarían las comunicaciones militares. Pero otra gran suceso de ese año fue la
inclusión del protocolo en el UNIX de la Universidad de Berkeley, el BSD 4.2.\par
Saltando tres años situándonos en 1986, la National Science Foundation construyo lo que serian los
cimientos de la internet que conocemos al financiar la construcción de una red para la conexión de
sus supercomputadoras. Esto gracias a la apertura de la NSF permitió a particulares conectarse a 
dicha red lo cual ayudo a su gran crecimiento, todo esto estaba y esta apoyado por el TCP/IP. Esta 
red existió hasta el año 1993, para ese entonces se era lo suficientemente maduro y rentable como 
para ser comercial y ya existían empresas dispuestas y con el conocimiento para llevarlo a cabo. Se 
puso en marcha un plan al año siguiente para reducir la influencia de la NFS y aumentar la 
rentabilidad de los incipientes ISP\footnote{Internet Service Provider} privados.\par
Mientras tanto el DoD y el gobierno norteamericano eligieron adoptar el modelo OSI y se pensó que
esto supondría el fin del modelo TCP/IP, incluso el gobierno llego a obligar su uso masivo, pero
TCP/IP siguió evolucionando sobre el fundamento de la practica y su calidad de estándar abierto.\par
Lo que nos puede dejar esta muy resumida historia del protocolo es que la internet no tiene un claro
inventor ni un destino claro, sino fue el trabajo en conjunto de varias personas e instituciones las
cuales querían solucionar sus problemas de comunicación. Fue gracias a las soluciones abiertas las
cuales permitió el mejoramiento por los usuarios, también los niveles de abstracción que se 
permitieron hizo realidad que varias tecnologías puedan convivir en armonía. Esta filosofía abierta
tan característica y que ha marcado tanto a la industria puede resumirse en el lema de la 
IETF\footnote{Internet Engineering Task Force}, expresado por David Clark: "Nosotros rechazamos 
reyes, presidentes y votaciones. Nosotros creemos en el consenso y en el código funcional".

\subsection{El Protocolo IP}
Este protocolo se encuentra en la segunda capa del modelo TCP/IP y la tercera del OSI. El se encarga
de transmitir los paquetes entre el origen y destino basándose en su dirección IP. Esto se logra
encapsulando los datos y formando su propio datagrama. Dentro se tienen dos partes principales el
encabezado y la carga, en el encabezado junto a otra metadata se coloca la dirección IP de origen y 
la de destino. El protocolo IP debe proveer de un servicio no orientado a la conexión no confiable y 
de mejor esfuerzo, también llamado servicio de datagramas. Esto quiere decir: no confiable, el 
protocolo no intenta recuperar los paquetes perdidos; no orientado a la conexión, cada paquete o 
datagrama es manejado independientemente IP desconoce si existe una secuencia lógica de envío; mejor
esfuerzo, IP no garantiza el servicio. Todo esto lleva a que otras capas se encarguen de la perdida
de los paquetes y solicitar su reenvío. IP también permite tres tipos de servicios: Unicast (uno a 
uno), Multicast (uno a varios) y Broadcast (uno a todos).\par
A continuación vemos la estructura de un datagrama de IP.
\begin{figure}[h!]
 \centering
 \includegraphics[width=1\textwidth]{ipDatagram.png}
 \caption[Paquete IP]{Paquete IP}
\end{figure}\par
Explicacion de los campos:\par
\begin{description}

\item[Version] representa la version del estándar actualmente es cuatro en transición a 6.
\item[Header Length] Tamaño del encabezado en múltiplos de 4 bytes.
\item[DS/ECN] Usado para especificar el nivel del servicio, actualmente no se usa y solo esta
presente para dar compatibilidad retroactiva.
\item[Identification] Identificación única del datagrama de un host, se incremente con cada 
datagrama.
\item[Flags] El primero siempre queda en cero, los otros dos indican si se fragmenta o tiene varios
fragmentos.
\item[TTL] Especifica la cantidad de caminos que puede tomar un paquete hasta que se lo descarte, se
utiliza para prevenir que el paquete quede para siempre dentro de un ciclo.
\item[Protocol] Especifica el numero del protocolo superior.
\item[Header Checksum] El checksum del header.
\item[Options] Restricciones de seguridad, registro de ruta(se agrega la dirección del router por el 
que pasa), marca de tiempo (por cada paso por router) y lista de rutas a pasar (por los únicos que 
puede pasar o por los que debe pasar entre otros)
\item[Padding] Se le agrega para asegurarse que tengo un tamaño de 4 bytes.

\end{description}


\section{El estándar IPv4}
\subsection{Aspectos Técnicos}
Ahora hablaremos de la version 4 del protocolo IP, la cual se implemento para el año 1981 y es 
actualmente el estándar global para internet.\par
El estándar IPv4 utiliza un direccionamiento de 32-bits, lo cual da una posibilidad de tan solo 
4.294.967.296 posibles direcciones, pero de esa problemática nos encargaremos mas adelante. Para 
anotar las direcciones de un método que nosotros los humanos podamos entender se utiliza un sistema 
de 4 octetos expresados en decimal y separados por un punto. Estas direcciones se dividen 
principalmente en dos partes. Una parte la primera es para identificar al ID de la red y la parte
restante identifica al host dentro de la red, ambas juntas como se podrá ver identifican 
unívocamente a un equipo conectado a la red. Historicamente se han dividio las direcciones en clases
dando a una asignacion llamada \textbf{direccionamiento con clases}. Se utilizan cinco tipos de 
clases \emph{A, B, C, D, E}
\begin{figure}[h!]
 \centering
 \includegraphics[width=1\textwidth]{clasesIP.png}
 \caption[Clases IP]{Clases IP}
\end{figure}\par
Como se puede apreciar la clase A permite tener muchos hosts por red pero solo puede haber pocas 
redes, por lo cual se utiliza para grandes redes. Además dentro de este rango entran dos direcciones 
reservadas la 0.0.0.0 y la 127.0.0.0 la cual asigna el enlace local, hace referencia a la propia 
interfaz de red, también existe un rango privado dentro de esta red el cual se extiende de 10.0.0.0 
a 10.255.255.255. Esta red puede mantener hasta 16.777.214 de hosts pero solo 128 redes distintas.
La clase B por su parte ya puede ofrecer mas redes por lo que su uso se da para redes de tamaño 
intermedio pudiendo dar cabida a 65.534 hosts pero con un mayor numero de redes posible el cual 
asciende a 16.384 posibles. A su vez el rango de direcciones privada se da entre 172.16.0.0 hasta 
172.31.255.255, también en esta clase se encuentra el link local. Por ultimo tenemos a la clase C la
que menor cantidad de hosts puede soportar, pero si pueden coexistir muchísimas redes distintas 
específicamente 2.097.152 y tan solo 254 hosts en cada red. Por eso esta clase se reserva para redes 
pequeñas. Su rango privado es familiar ya que la gran mayoría de routers hogareños y para pequeñas 
empresas lo utilizan van desde 192.168.0.0 hasta 192.168.255.255. Las dos clases siguientes D y E se
destinan para la multidifusi\'on y para usos futuros respectivamente.\par
Dado que este sistema mostró ciertas falencias para el rápido crecimiento de las conexiones y hosts
necesarios, se desarrollo las subredes y las mascaras de subred. La solución a este problema es 
permitir la división de una red en varias partes para uso interno, pero aún actuar como una sola red 
ante el mundo exterior. Esto se implementa con una mascara la cual tiene una estructura parecida a
la de una dirección IP, un ejemplo seria una mascara donde no existe división la cual quedaría como 
255.255.255.0, solamente el ultimo octeto queda libre para variar seria una mascara para clase C. 
Entonces una mascara de clase A seria 255.0.0.0 y por ulitmo una de clase B se expresa 25.255.0.0. 
La mascara permite poder identificar a la subred dentro de la red principal,pero sin que el exterior 
note la división, por lo cual no es necesario solicitar nuevas direcciones. Por Ejemplo, al 
introducirse subredes, se cambian las tablas de enrutamiento, agregando entradas con forma de (esta 
red, subred, 0) y (esta red, esta subred, host). Por lo tanto, un enrutador de la subred k sabe cómo 
llegar a todas las demás subredes y a todos los hosts de la subred k; no tiene que saber los 
detalles sobre los hosts de otras subredes. De hecho, todo lo que se necesita es hacer que cada 
enrutador haga un AND booleano con la máscara de subred de la red para deshacerse del número de host 
y buscar la dirección resultante en sus tablas (tras determinar de qué clase de red se trata). Por 
eso se puede afirmar que la división de redes reduce espacio en la tabla de enrutamiento creando una 
jerarquía de tres niveles, que consiste en red, subred y host.


\subsection{Porque abandonar IPv4}
Uno de los mayores problemas que presenta IPv4 es su tamaño, el cual ya ha quedado chico para los
estándares actuales. Este es un numero de solo 32bits con lo cual la cantidad máxima de números
posibles es $2^{32}$ cual da un total de 4.294.967.296 posibles dirección eso es menos de una
dirección por habitante\footnote{población actual de 7 mil millones de personas}.\par
Esto se ve mas limitante con la llegada masiva de los dispositivos mobiles, los cuales se conectan
a internet y la entrada de los grandes mercados emergentes del BRIC (Brasil, Rusia, India, China).
 
\section{La llegada de IPv6}
Principalmente internet era un prototipo que se hizo con un protocolo sencillo, nunca se creyó que 
tuviese un crecimiento tan descomunal como para que en el futuro surgieran problemas tales como la 
seguridad entre el tráfico de datos a través de la red, y la implementación de internet en otras 
tecnologías como por ejemplo celulares. Los problemas que hicieron inadecuada la version 4 de IP 
hicieron que se cambiara el sistema por completo por lo tanto los desarrolladores fueron forzados a 
crear un protocolo que resolviera los problemas actuales y trabajara diligentemente para asegurar 
que estos problemas no se encontraran nunca mas. Los desarrolladores del protocolo trabajaron diez 
años pero su implementación se hizo recientemente, es la implementación de un nuevo protocolo, 
escalable, ilimitado y con una gran proyección a futuro.\par

\section{Los Beneficios de IPv6}
Entre los principales beneficios de IPv6 fue la solución a dos problemas fundamentales que tenia su 
protocolo antecesor: la falta de direcciones, y la escalabilidad de la ruta. Las soluciones 
implementadas fueron:
\begin{itemize}
\item Incremento de las direcciones IP
\item Desarrollo en la jerarquía de direcciones
\item Organización en los perfiles de red
\item Autoconfiguracion sencilla de la red
\item Mejora en la escalabilidad del enrutamiento
\item Enrutamiento a la mejor dirección posible(Anycast)
\item Mejoras en la seguridad
\item Mejoras en la movilidad
\item Se agregan nuevas tecnologías que mejoran la performance del protocolo

\end {itemize}

\begin{figure}[h!]
 \centering
 \includegraphics[width=0.8\textwidth]{comparativo_ip.png}
\caption[comparacion versiones IP]{Cuadro comparativo de las direcciones IP}
\end{figure} \par

\subsection{Jerarquía de Direcciones}
IPv6 divide las direcciones en distintos ámbitos definidos o limites en los cuales se delegan las 
direcciones, por lo que soluciona problemas de disgregación. Cuenta con un sistema de enrutamiento 
que permite discernir rápidamente el tipo de paquete que es, así el enrutamiento se realiza de una 
forma mas sencilla y eficiente.
Otra de las herramientas que cabe destacar en IPv6 es el agregador de nivel superior \textbf{Top 
Level Aggregator (TLA)} que consta de dos propósitos: el primero es la designación de un gran bloque 
de direcciones compuesto por pequeños bloques que tienen la función de dar una conectividad 
descendiente a los que necesitan acceso a la red. El segundo es la detección de la ruta de origen, 
resulta mas claro ver quienes son los que tienen bloques de direcciones de proveedores y quienes son 
los que tienen una dirección local. Esto aumenta la eficiencia del núcleo de internet porque al 
agruparse de esta forma y haber una mejor organización de las direcciones, es mucho mas claro el 
pasaje de información.\par
Otro de los organismos existentes en este protocolo es el \textbf{NLA Next Level Aggregator} que si 
bien es demasiado pequeño para ser clasificado como un TDA, tiene una cadena principal regional 
extensa y cuenta con un numero de pequeños clientes. Sirve para dentro del gran bloque que 
proporciona el TDA, romper su porción de direcciones y delegar las direcciones obtenidas a los 
\textbf{Sitios Agregadores de Servicio SLA}. Estos SLA se encargan de identificar subnets que no 
estén en el sitio.Otro beneficio importante que se puede destacar del NLA se relaciona con la 
estabilidad de ruta actual de todas las rutas a nivel mundial dejando de lado las inestabilidades 
conocidas como cortes, fallo en enlaces y lentitud en el tráfico de datos. Debido a esta 
inestabilidad, el concepto de "route dampening" surgió y trabaja de la siguiente forma: cada vez que 
una ruta se retira y se renuncia, se le asigna una penalización que se mantiene hasta el lugar donde 
se encuentra la inestabilidad que por lo general suele ser una frontera exterior o  protocolo de 
puerta de enlace. Mientras mas alta sea la inestabilidad, mas alta será la penalidad asociada con la 
ruta. Cuando esa penalización alcanza cierto nivel, la ruta se retira y debe someterse a un período 
de espera. Mientras mas tiempo pasa, la pena va decreciendo hasta que la misma se disuelve y vuelve 
a estar permitida y se la reinserta en la tabla  de BGP del router. El propósito de esto es proveer 
una forma de negociar las inestabilidades de una forma tal que se minimice el costo de otros 
procesos que son importantes.\par
Cabe destacar que otro beneficio importante de la agregación la mejora importante en el 
enrutamiento, que forma parte de un requisito fundamental en IPv6. Así como hay mas direcciones 
también hay extensas ramificaciones y formas de organizarlas y controlarlas para que la asignación 
de las mismas  no sea un problema.

\subsection{Un mecanismo de direccionamiento  mas sencillo}
El modelo IPv6 esta  caracterizado por tener  una dirección de 128.Los primeros 64 bits están 
destinados a la numeración de red y los últimos 64 se utilizan para la numeración del host, al tener 
128 bits es mucho mas variada la cantidad de direcciones que se pueden formar y también aumenta la 
seguridad de la red. Debemos recordar que los últimos 64 bits del id del host se obtienen a partir 
de las direcciones MAC de la red de interfaz. Por convención la primera dirección se da normalmente 
al enrutador designado y el resto de las direcciones se asignan a los host de la subred con el 
ultimo domicilio. En IPv6 es un tanto diferente ya que  sabemos que la id es una dirección  de 64 
bits que se obtiene de la dirección MAC. Hoy en día las direcciones MAC son de 48 bits, para llegar 
a 64 bits se las rellena con cadenas cadenas 0xff y 0xFE (: FF: FE: en términos IPv6) para cubrir la 
diferencia que existe entre la dirección MAC de la identificación de la compañía y el ID 
proporcionado por el proveedor de la MAC.\par
El conflicto reside en si es necesario que las direcciones MAC tengan que cambiar su longitud  solo 
porque con el protocolo IPv6 cambia las longitudes del direccionamiento, si la necesidad de 
implementar direcciones MAC tan largas, la siguiente opción para la longitud proporcionara mas de 
1.8E019 direcciones MAC en lugar de usar (264-248) si este viene a ser el
caso, simplemente puede dejar el relleno de la dirección MAC, y el uso de los 64 completo
bits de la dirección MAC para el ID de host.

\subsection{Autoconfiguracion de direcciones}
La mejor ventaja de IPv6 sin dudas es la capacidad de autoconfiguracion de las direcciones IP. Antes 
de entrar en detalles sobre el tema hablaremos de \textbf{la dirección de multidifusion}. Esta 
consiste en una dirección multicast que se puede asignar simultáneamente a mas de una maquina.Esta 
dirección envía los paquetes al grupo de equipos asignados a la misma.Todas las maquinas que están 
asignadas a esa dirección se dice que están en un grupo multicast, cuya dirección es la dirección  
de multidifusion que utilizan.Los equipos conectados envían y reciben datos desde mas de un host. Se 
utiliza este estilo para realizar 1 a N o M transacciones.\par
Asociando este concepto con el de autoconfiguracion, se puede decir que una red lan es un grupo de 
maquinas y que cuando una nueva maquina ingresa a ese grupo, esta conectada y usan IPv6, se le 
enviara a esa maquina nueva un paquete de multidifusion, este paquete se destinara a la dirección 
de un ámbito local. Cuando el router ve que este paquete entra, este se puede responder con la 
dirección de red de la maquina nueva. La respuesta recibe el paquete y a su vez, lee el numero de 
red que el router tiene enviado. Si se asigna una dirección de IPv6 añadiendo su ID de host. Todo 
este proceso no requiere ninguna intervención manual por parte del administrador, esto asegura 
también la unicidad de la dirección, la maquina esta garantizada para tener la dirección única, 
porque el numero de red exclusivamente asignado por el numero de router de esa red.
Este mecanismo ahorra al usuario un montón de problemas tales como la configuración manual cuando el 
equipo se mueve de una red a otra, la realización de un seguimiento de las direcciones que se ha 
asignado y cuales están libres en un tiempo dado.
Esto le da una facilidad muy grande al administrador de redes ya que no debe perder tiempo en tareas 
de seguimiento y en el renombramiento de la red.

\subsection{Mejora en la Escalabilidad del Enrutamiento Multicast}
En este apartado ampliaremos mas el concepto de dirección de multidifusion: En los inicios de internet, los 
problemas de congestión fueron tolerados ya que los datos que enviaban y recibían los usuarios no tenían porque 
ser en tiempo real. Hoy, por el contrario, las empresas están utilizando internet para una amplia gama de 
aplicaciones. Esta tendencia de tener en nuestros dispositivos de información tal como el estado del tiempo , 
la 
cotización de la bolsa y las noticias del día entre otros, hace que se provoque la necesidad de un nuevo tipo 
de envío de trafico en el cual se pueda enviar 
un conjunto de datos a muchas personas. Antiguamente se copiaba el archivo a enviar tantas veces como 
destinatarios existían y se los enviaba a cada uno.El problema con este método es cuando se quiere difundir un 
archivo en tiempo real de forma masiva, el ancho de banda quizás no sea lo suficientemente grande como para 
manejar tal velocidad de datos.La idea de la multidifusion justamente es tomar ese conjunto de datos y enviarlo 
por una dirección de multidifusion, un grupo colectivo de direcciones por ejemplo en IPv4 se utiliza el 
intervalo de 
224.0.0.0 a 239.255.255.255. Cuando se quiere enviar algo a múltiples destinatarios, se asigna una dirección 
temporal del intervalo para escuchar los paquetes que llegan del origen. Este concepto nos ayuda a ahorrar 
mucho 
ancho de banda y velocidad de datos además de mantener una estructura de enrutamiento eficaz.\par
En IPv6 es posible asignar ciertas secuencias de multidifusion para ser transferido dentro de un área 
determinada y no permitir que los paquetes salgan de esa zona, los limites de envío y recepción de paquetes 
serán bien conocidos y entendidos por todos.

\begin{figure}[h!]
 \centering
 \includegraphics[width=0.7\textwidth]{formato_multicast.png}
 \caption[Formato Multicast]{Formato de una dirección multicast en IPV6}
\end{figure} \par

En la figura de arriba podemos ver el Formato de una dirección multicast. Todos los ocho bits se establecen en 
uno, lo que permitirá a un dispositivo de enrutamiento saber inmediatamente si el paquete es de multidifusion y 
sujeto a la manipulación asociado con su tipo. Los siguientes cuatro bits se utilizan para los "flags": los 
tres 
primeros flags son reservados y no están definidos por lo que están establecidos en cero, el cuarto bit se lo 
conoce como bit T y se utiliza para decidir si la dirección multicast es una dirección permanente o una 
asignación temporal como se ha hablado en el párrafo anterior. Ese ultimo campo nos informara si la dirección 
de multidifusion que se esta utilizando es la que viene de serie. El siguiente campo nos dirá hasta que punto 
puede llegar la multidifusion, en lo que áreas de un dominio de enrutamiento de un paquete puede viajar y la 
dirección del grupo que puede ser alcanzado. Toma los siguientes valores:\par
%ver al final como se puede emprolijar esta imagen
\begin{figure}[h!]
 \centering
 \includegraphics[width=0.8\textwidth]{valores.png}
 \caption[Valores del campo scope]{Valores que puede tomar el campo "scope"}
\end{figure} \par

Dependiendo de como asignemos nuestra dirección multicast, podremos controlar que tan lejos viajarán los 
paquetes. Ejemplo; no es lo mismo la difusión masiva de un vídeo a una cantidad muy grande de usuarios que la 
difusión de un ppt a un grupo de trabajo que se encuentra dentro de la misma red LAN. Esto significa que se 
controlará la propagación de información y que información será propagada y a quien se la enviará, por lo 
tanto, en lugar de dejarle a un administrador de red la tarea de colocar filtros para que algunos paquetes le 
lleguen a cierta gente y a otros no, podemos dejarle esta tarea a este mecanismo. Esto también permite una 
fácil configuración de un nivel de privacidad y muy bien definido, que también facilita el mantenimiento del 
mismo.
\subsection{Dirección anycast}


\section{Implementar IPv6}
\section{Desarrollo a futuro}

\newpage
\begin{thebibliography}{9}

\bibitem{redesTenem}
  Andrew S. Tanenbaum
  \emph{Redes de Computadoras}.
  USA,
  2003.

\bibitem{TaoIETF}
  Paul Hoffman
  \emph{The Tao of IETF: A Novice's Guide to the Internet Engineering Task Force}.
  USA,
  2012.
		
\bibitem{historyTCPiP}
  Gary C. Kessler
  \emph{An Overview of TCP/IP Protocols and the Internet}.
  USA,
  9 Nov 2010.
		
\bibitem{poolIPv4}
 nro.net
 \emph{Free Pool of IPv4 Address Space Depleted.} 
 19 Mar 2013\\
	\url{https://www.nro.net/news/ipv4-free-pool-depleted}	

\bibitem{IPv6 Architecture}
 cu.ipv6ft.org 
 \emph{Introduction to IPv6 Architecture.} 
 26 Mar 2013\\
	\url{http://www.cu.ipv6tf.org/literatura/sample.pdf}
					
\end{thebibliography}

\end{document}
