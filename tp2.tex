\documentclass[12pt,a4paper]{article}
\usepackage[utf8]{inputenc}
\usepackage[spanish]{babel}
%\usepackage{amsmath}
%\usepackage{amsfonts}
%\usepackage{amssymb}
\usepackage{hyperref}
\usepackage[pdftex]{graphicx}
\usepackage{fancyhdr}
\usepackage[font=small,labelfont=bf]{caption}
\pagestyle{fancy}
\lhead{\bfseries Redes de Datos -- Cloud Computing}
\rhead{}
%\chead{}
\newcommand{\HRule}{\rule{\linewidth}{0.5mm}}
\usepackage[left=2cm,right=2cm,top=2cm,bottom=2cm]{geometry}
\author{Ignacio Perez Laborda}

\hypersetup{pdfborder = {0 0 0}}

\begin{document}

\begin{titlepage}

\begin{center}

% Upper part of the page
\includegraphics[width=0.25\textwidth]{./logo_UB.png}\\[1cm] 

\textsc{\LARGE Universidad de Belgrano}\\[1.5cm]

\textsc{\Large Redes De Datos}\\[0.5cm]

%title
\HRule \\[0.4cm]
{ \huge \bfseries Trabajo Practico 2 -- Cloud Computing}\\[0.4cm]
\HRule \\[1.5cm]

% Author and supervisor
\begin{minipage}{0.4\textwidth}
\begin{flushleft} \large
\emph{Alumno:}\\
Ignacio \textsc{P\'erez Laborda}\\
Barbara \textsc{Mart\'inez}\\
\end{flushleft}
\end{minipage}
\begin{minipage}{0.4\textwidth}
\begin{flushright} \large
\emph{Matricula:} \\
502--10426\\
502--10402\\
\end{flushright}
\end{minipage}\\[1.5cm]

\vfill

%Bottom of the page
{\large \today}

\end{center}

\end{titlepage}


\tableofcontents

\listoffigures

\newpage

\section{Que es Cloud Computing?}
Esta es una prueba de letra común \textbf{Negrita} \emph{Emfasis}

\section{Distintos Servicios de Cloud Computing}
\subsection{SaaS -- Software as a Service}
\subsection{PaaS -- Platform as a service}
\subsection{IaaS -- Infrastructure as a service}

\section{Aplicaciones Practicas}

\section{Cloud Computing para usuarios finales}

\section{Conclusiones}

\newpage
\begin{thebibliography}{9}

\bibitem{wnaReport1}
  WNA
  \emph{Comparison of Lifecycle Greenhouse Gas Emissions of Various Electricity Generation Sources}.
  London, UK,
  2011.

\bibitem{espTempEins}
  Rafael Ferraro
  \emph{El espacio-tiempo de Einstein.}
  Buenos Aires, Argentina,
  2008.
		
\end{thebibliography}

\end{document}
